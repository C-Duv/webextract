\documentclass{article}
\usepackage[frenchb]{babel}
\usepackage[T1]{fontenc}
\usepackage{times}
\usepackage[utf8]{inputenc}
\usepackage[titletoc,toc,title]{appendix}

\title{\emph{Object search}}
\author{Gautier DI FOLCO}
\date{Janvier 2014}

\begin{document}
\maketitle
\tableofcontents

\begin{abstract}
Le but est de voir s'il est possible, à partir de n'importe quel site marchand,
via \emph{machine learning} d'en extraire les informations principales.
\end{abstract}

\section{Méthodologie}
Nous sommes partie sur une approche CRF et nous avons pour cela pris chaque noeud
d'un document HTML (en ayant filtré un grand nombre de noeuds non-suceptible de
nous intéresser pour les informations que nous cherchons, afin de limiter le nombre de
noeuds inintéressant, qui rendraient l'apprentissage plus long) et nous avons établi
une liste de \emph{feature}s.

Nous avons manuellement marqué (via l'ajout d'un attribut \emph{data-tess-label})
une partie (10 pages) de notre jeu de données (il comporte 50 pages par site,
ces sites étant amazon.fr carrefour, ldlc, fnac et rueducommerce) afin de nous
en servir comme base d'apprentissage pour notre logiciel de CRF (\emph{CRFSuite}).

\subsection{Features extraites}

\begin{itemize}
    \item[parProdClass] Le nombre de parents ayant une classe contenant le mot "prod"
    \item[ancProdClass] Le nombre d'ancêtres ayant une classe contenant le mot "prod"
    \item[ancDescClass] Le nombre d'ancêtres ayant une classe contenant le mot "desc"
    \item[selfProdClass] Le nombre de classes ayant "prod" dans son nom pour le noeud courant
    \item[selfDescClass] Le nombre de classes ayant "desc" dans son nom pour le noeud courant
    \item[selfCurr] Le nombre d'occurences d'un symbole de monnaie dans le texte du noeud courant
    \item[contDesc] Le nombre d'occurences de "desc" dans le texte du noeud courant
    \item[selfEl] Le nom du noeud courant
    \item[parEl] Le nom du noeud parent
    \item[selfIP] Le contenu de l'attribut \emph{itemprop}
    \item[selfClass] La classe courante
    \item[parClass] La classe du noeud parent
    \item[selfDepth] Le nombre d'ancêtres
    \item[selfChilds] Le nombre de déscendants
    \item[inHn] Le noeud courant ou un de ses ancêtre est-il un noeud de type h1, h2 ou h3
\end{itemize}

\subsection{Données recherchées}
Nous avons tenté de chercher la description, le titre et le prix de chaque page.

Dans cet objectif, notre convertisseur de HTML en \emph{dataset CRFSuite} va, pour chaque
noeuds non-filtré de la page, chercher si ce noeud a l'attribut \emph{data-tess-label}
si c'est le cas, le type de l'enregistrement aura cette valeur, si non il aura comme
type "other".

\subsection{Méthodologie}
Dans un premier temps nous récupérons un jeu de 50 pages pour 5 sites, nous en
marquons manuellement 10.

Nous convertissons ensuite ces deux jeux (l'original de 50 pages et le marqué de 10
pages) en format CRFSuite.

Puis nous lançons la procédure d'apprentissage, une fois les modèles générés
nous les appliquons sur les pages non-marquées au format CRFSuite.

Ensuite nous convertissons le résultat en page HTML.

Nous récupérons de plus des statistiques sur les phases d'apprentissage et de
marquage.

\subsubsection{Première tentative : inter-site}
Lors de l'apprentissage nous avons générés les 120 combinaisons possibles de sites
(afin de voir si l'ordre des jeux de données influence l'apprentissage)
puis nous avons retirer un jeu (le dernier site de la liste) et nous avons fait
un apprentissage sur les jeux restants.

Puis nous appliquons chaque modèles aux pages du jeu restant.

Nos observations sont les suivantes :

\begin{itemize}
 \item L'ordre des jeux de données à une influence sur le temps d'apprentissage
 \item L'apprentissage est très long (entre 20h et 6 jours)
 \item Les modèles générés sont inefficaces (aucun marquage n'a remonté une information recherchée)
\end{itemize}

Nous en concluons qu'il faut, avant de poursuivre les investigations plus loin,
valider que notre utilisation de CRFSuite est correcte.

\subsubsection{Deuxième tentative : site par site}
Les pages de chaque site étant générés dynamiquement à partir d'un gabarit fixe
CRFSuite devrait pouvoir s'y retrouver plus facilement si chaque site à un apprentissage
et un marquage isolé.

Malheureusement ce n'est pas le cas, les temps d'apprentissages sont encore long (de
quelques heures à quelques jours) et les marquages sont encore inefficaces.

Nous en déduisons donc que soit le nombre de noeuds "other" est trop important par
rapport aux noeuds intéressants, soit les features sont mal choisies/insuffisantes.

\subsubsection{Troisième tentative : évaluation des \emph{feature}s}
Le but est de sélectionner les \emph{feature}s les plus discriminantes.

$Soit \Phi_i une \emph{feature}.$.

Pour $val \in \{titre, description, prix\}$.

Pour $i = 1..n$ où $n$ est le nombre de \emph{feature}s.

Soit \[  \Delta^{val}_{\Phi_i}= \frac{|moyenneValeurs(\Phi_i), elementsP) - moyenneValeurs(\Phi_i), elementsN)|}{maxValeur(\Phi_i) - minValeur(\Phi_i)}\]
où $elementsP$ est l'ensemble des noeuds marqués par $val$ et $elementsN$ les autres.

On choisit les $K$ \emph{feature}s les plus driscriminantes (qui ont un $\Delta^{val}_{\Phi_i}$ le plus proche de 1 possible).

On va ensuite chercher une valeur seuille pour laquelle le discriminant est pertinent.

Voici pour chaque site les résultats :

\paragraph{amazon}
Voici les \emph{feature}s par ordre décroissant de pertinence :
\begin{itemize}
    \item{title} inHn=0.9815057868989381 selfDepth=0.15146107124391342 ancProdClass=0.03328958358191147
    \item{price} ancProdClass=0.9667104164180885 selfDepth=0.10568178589894371 selfCurr=0.0189714831165732
    \item{description} selfProdClass=0.990096647178141 selfDepth=0.23241345219629436 ancProdClass=0.03328958358191147
\end{itemize}

Les \emph{feature}s sélectionnées pour la phase d'apprentissage sont donc :
inHn selfDepth ancProdClass selfCurr selfProdClass

\paragraph{carrefour}
Voici les \emph{feature}s par ordre décroissant de pertinence :
\begin{itemize}
    \item{title} inHn=0.962248322147651 ancProdClass=0.9253355704697986 selfDepth=0.12138213087248317
    \item{img} ancProdClass=0.07466442953020135 selfDepth=0.058882130872483174 inHn=0.037751677852348994
    \item{price} selfDepth=0.25263213087248315 ancProdClass=0.07466442953020135 inHn=0.037751677852348994
    \item{description} ancProdClass=0.07466442953020135 inHn=0.037751677852348994 ancDescClass=0.03271812080536913
\end{itemize}

Les \emph{feature}s sélectionnées pour la phase d'apprentissage sont donc :
inHn ancProdClass selfDepth ancDescClass

\paragraph{fnac}
Voici les \emph{feature}s par ordre décroissant de pertinence :
\begin{itemize}
    \item{img} selfDepth=0.1394904349903582 inHn=0.047577603713373946 ancProdClass=0.035030461270670145
    \item{title} inHn=0.9524223962866261 selfDepth=0.11810006065880743 ancProdClass=0.035030461270670145
    \item{price} selfDepth=0.12879524782458282 inHn=0.047577603713373946 ancProdClass=0.035030461270670145
    \item{description} selfDepth=0.29457064889410156 inHn=0.047577603713373946 ancProdClass=0.035030461270670145
\end{itemize}

Les \emph{feature}s sélectionnées pour la phase d'apprentissage sont donc :
selfDepth inHn ancProdClass

\paragraph{ldlc}
Voici les \emph{feature}s par ordre décroissant de pertinence :
\begin{itemize}
    \item{price} selfCurr=0.9768084779603747 parProdClass=0.14252802948855783 selfDepth=0.10417946551988944
    \item{title} inHn=0.980187375211181 parProdClass=0.14252802948855783 selfDepth=0.08332053448011056
    \item{img} selfDepth=0.14582053448011056 parProdClass=0.14252802948855783 ancProdClass=0.06839707162238266
    \item{description} selfDescClass=0.9906312394409461 ancDescClass=0.6569395382173757 parProdClass=0.14252802948855783
\end{itemize}

Les \emph{feature}s sélectionnées pour la phase d'apprentissage sont donc :
selfCurr parProdClass selfDepth inHn ancProdClass selfDescClass ancDescClass

\paragraph{rueducommerce}
Voici les \emph{feature}s par ordre décroissant de pertinence :
\begin{itemize}
    \item{img} selfDepth=0.13216042606821976 selfCurr=0.0349432857665569 ancDescClass=0.023783388218075376
    \item{title} inHn=0.9853640687888767 selfCurr=0.0349432857665569 ancDescClass=0.023783388218075376
    \item{price} selfDepth=0.25672846282066913 selfCurr=0.0349432857665569 ancDescClass=0.023783388218075376
    \item{description} ancDescClass=0.9762166117819246 selfDepth=0.1877159816237753 selfCurr=0.0349432857665569
\end{itemize}

Les \emph{feature}s sélectionnées pour la phase d'apprentissage sont donc :
selfDepth selfCurr ancDescClass inHn

Les apprentissages ont été lancés sur ces jeux de données "restreints", comme nous
pouvions nous y attendre, le temps d'apprentissage a été grandement diminué (puisqu'il
ne dure plus que quelques secondes).

En revanche le résultat du marquage est toujours le même : aucun noeud recherché n'est
marqué.

\subsubsection{Quatrième tentative : ajout de nouvelles \emph{feature}s}
Partant du constat que les \emph{feature}s les moins pertinentes ont été retirées
et que malgré cela les résultats ne sont pas satisfaisants, nous en avons créé de
nouvelles (62) ayant un taux de discrimination plus élevé.

Voici les nouveaux niveau de discrimination :

\paragraph{amazon}
Voici les \emph{feature}s par ordre décroissant de pertinence :
\begin{itemize}
    \item{title} inHn=0.9815057868989381 ancTitleClass=0.9152845722467486 ancTitleId=0.6935568547905977
    \item{price} parPriceId=0.9923636797518196 selfPriceId=0.9917670922324305 ancProdClass=0.9667104164180885
    \item{description} ancDescId=0.9909318697052858 selfDescClass=0.9902159646820189 selfProdClass=0.990096647178141
\end{itemize}

Les \emph{feature}s sélectionnées pour la phase d'apprentissage sont donc :
inHn ancTitleClass ancTitleId parPriceId selfPriceId ancProdClass ancDescId selfDescClass selfProdClass

\paragraph{carrefour}
Voici les \emph{feature}s par ordre décroissant de pertinence :
\begin{itemize}
    \item{title} ipNameChild=1.0 ipName=1.0 ancTitleClass=0.9874161073825504
    \item{img} nbChildGallery=0.9924496644295302 selfGalleryClass=0.9924496644295302 parPhotoClass=0.9840604026845637
    \item{price} sibPriceClass=0.9739932885906041 ipPriChild=0.8991610738255034 inBuy=0.43833892617449666
    \item{description} sibTitleClass=0.09228187919463088 sibZoomClass=0.09144295302013423 sibPhotoClass=0.08389261744966443
\end{itemize}

Les \emph{feature}s sélectionnées pour la phase d'apprentissage sont donc :
ipNameChild ipName ancTitleClass nbChildGallery selfGalleryClass parPhotoClass sibPriceClass ipPriChild inBuy sibTitleClass sibZoomClass sibPhotoClass

\paragraph{fnac}
Voici les \emph{feature}s par ordre décroissant de pertinence :
\begin{itemize}
    \item{img} ipImageChild=0.9966637655932695 nbChildVisual=0.9926022628372497 selfVisualClass=0.9926022628372497
    \item{title} ipName=1.0 ipNameChild=0.9965187119234117 inHn=0.9524223962866261
    \item{price} selfPriceClass=0.9666376559326951 parPriClass=0.9506817522483318 ancPriceClass=0.4348709022338265
    \item{description} ipDescChild=0.539217237650658 selfDepth=0.29457064889410156 ancPriceClass=0.06512909776617348
\end{itemize}

Les \emph{feature}s sélectionnées pour la phase d'apprentissage sont donc :
ipImageChild nbChildVisual selfVisualClass ipName ipNameChild inHn selfPriceClass parPriClass ancPriceClass ipDescChild selfDepth

\paragraph{ldlc}
Voici les \emph{feature}s par ordre décroissant de pertinence :
\begin{itemize}
    \item{price} selfPriceClass=0.979265857779143 selfCurr=0.9768084779603747 parPriClass=0.9562279219781907
    \item{title} ipName=1.0 ipNameChild=0.9984641376132698 inHn=0.980187375211181
    \item{img} nbChildPhoto=0.9984641376132698 ipImageChild=0.9984641376132698 selfProdId=0.9138381201044387
    \item{description} ipDescChild=0.9938565504530794 selfDescClass=0.9890953770542159 ancDescClass=0.6538678134439154
\end{itemize}

Les \emph{feature}s sélectionnées pour la phase d'apprentissage sont donc :
selfPriceClass selfCurr parPriClass ipName ipNameChild inHn nbChildPhoto ipImageChild selfProdId ipDescChild selfDescClass ancDescClass

\paragraph{rueducommerce}
Voici les \emph{feature}s par ordre décroissant de pertinence :
\begin{itemize}
    \item{img} nbChildPhoto=0.9963410171972191 childPhotoClass=0.49451152579582874 ancProdClass=0.2607391145261617
    \item{title} inHn=0.9853640687888767 selfProdClass=0.9259055982436882 parProdClass=0.8816319063300403
    \item{price} ancPriceClass=0.9619465788510794 selfPriceClass=0.9527991218441273 parPriClass=0.950969630442737
    \item{description} ipDescChild=0.9963410171972191 ancDescClass=0.9721917306988657 parProdClass=0.8816319063300403
\end{itemize}

Les \emph{feature}s sélectionnées pour la phase d'apprentissage sont donc :
nbChildPhoto childPhotoClass ancProdClass inHn selfProdClass parProdClass ancPriceClass selfPriceClass parPriClass ipDescChild ancDescClass

Voici le descriptif de chacune d'entre elle :
\begin{itemize}
    \item[ancDescId] Quel est le nombre d'ancêtres qui possèdent "desc" dans leur attribut id
    \item[ancPriceClass] Quel est le nombre d'ancêtres qui possèdent "price" dans leur attribut class
    \item[ancTitleClass] Quel est le nombre d'ancêtres qui possèdent "title" dans leur attribut class
    \item[ancTitleId] Quel est le nombre d'ancêtres qui possèdent "title" dans leur attribut id
    \item[childPhotoClass] Quel est le nombre d'enfants qui possèdent "photo" dans leur attribut class
    \item[inBuy] Quel est le nombre d'ancêtre et du noeud courant qui on une classe contenant "buy"
    \item[ipDescChild] Le noeud courant contient-il la chaine "desc" dans son attribut itemprop
    \item[ipImageChild] Le noeud courant contient-il la chaine "image" dans son attribut itemprop
    \item[ipName] Le noeud courant contient-il la chaine "name" dans son attribut itemprop
    \item[ipNameChild] Quel est le nombre de jeu enfants qui contiennent la chaine "name" dans leur attribut itemprop
    \item[ipPriChild] Le noeud courant contient-il la chaine "pri" dans son attribut itemprop
    \item[nbChildGallery] Quel est le nombre de noeuds fils qui contiennent la chaine "gallery" dans leur attribut class
    \item[nbChildPhoto] Quel est le nombre de noeuds fils qui contiennent la chaine "photo" dans leur attribut class
    \item[nbChildVisual] Quel est le nombre de noeuds fils qui contiennent la chaine "visual" dans leur attribut class
    \item[parPhotoClass] Le noeud parent contient-il la chaine "photo" dans son attribut class
    \item[parPriceId] Le noeud parent contient-il la chaine "price" dans son attribut id
    \item[parPriClass] Le noeud parent contient-il la chaine "pri" dans son attribut class
    \item[selfDescClass] Le noeud courant contient-il la chaine "desc" dans son attribut class
    \item[selfGalleryClass] Le noeud courant contient-il la chaine "gallery" dans son attribut class
    \item[selfPriceClass] Le noeud courant contient-il la chaine "price" dans son attribut class
    \item[selfPriceId] Le noeud courant contient-il la chaine "price" dans son attribut id
    \item[selfProdId] Le noeud courant contient-il la chaine "prod" dans son attribut id
    \item[selfVisualClass] Le noeud courant contient-il la chaine "visual" dans son attribut class
    \item[sibPhotoClass] Quel est le nombre de noeuds voisins qui contiennent la chaine "photo" dans leur attribut class
    \item[sibPriceClass] Quel est le nombre de noeuds voisins qui contiennent la chaine "price" dans leur attribut class
    \item[sibTitleClass] Quel est le nombre de noeuds voisins qui contiennent la chaine "title" dans leur attribut class
    \item[sibZoomClass] Quel est le nombre de noeuds voisins qui contiennent la chaine "zoom" dans leur attribut class
\end{itemize}

Malheureusement la phase d'apprentissage et de marquage ne sont pas plus pertinentes.

\section{Conclusions}
Nous avons vu que nos tentatives successives n'ont pas données de résultats satisfaisants.

Cela peut-être du à :
\begin{enumerate}
    \item Une mauvaise utilisation de CRFSuite
    \item L'approche CRF n'est pas adapté à notre cas/nos features
    \item Mauvais relevé des features
    \item Taux de noeuds négatif trop important
\end{enumerate}

Notre étude ne nous a cependant pas permit de determiné la faisabilité ou la non-faisabilité
d'une extraction de produits depuis un page produit d'un site de e-commerce quelconque par
apprentissage.

Notre étude à simplement mis en avant le fait que notre méthodologie est incorrecte pour
arriver à cette fin.

Un changement d'approche pourrait aussi être envisagé : au lieu de prendre la structure
HTML des pages comme source, prendre la représentation visuelle (les éléments visuels
mis en avant).

\end{document}
